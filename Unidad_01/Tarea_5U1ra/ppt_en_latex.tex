\documentclass{beamer}
\usepackage[utf8]{inputenc}
\usepackage[spanish]{babel}
\usepackage{graphicx}
\usepackage{amsmath}

\usetheme{default}
\setbeamertemplate{navigation symbols}{}

\title{Uso de Programación Lineal en Redes de Salud}
\author{ Torres Cruz Fred \\ Ccora Quispe Kenny Leonel \\ Universidad Nacional del Altiplano - FINESI}

\begin{document}

\begin{frame}
  \titlepage
\end{frame}

\begin{frame}{1. ¿Cuál era el problema?}
\begin{itemize}
  \item En Brasil, muchas personas con cáncer tenían que viajar lejos para recibir tratamiento.
  \item Eso generaba cansancio, gastos y molestias para los pacientes.
  \item Los investigadores querían encontrar una forma de reducir esas distancias.
\end{itemize}
\end{frame}

\begin{frame}{2. ¿Qué hicieron los científicos?}
\begin{itemize}
  \item Aplicaron una técnica matemática llamada \textbf{programación lineal}.
  \item Esta herramienta sirve para encontrar la mejor solución bajo ciertas condiciones.
  \item La aplicaron en hospitales de la red UNACON-RS en el estado de Río Grande do Sul.
\end{itemize}
\end{frame}

\begin{frame}{3. ¿Cómo aplicaron la programación lineal?}
\begin{itemize}
  \item Definieron la función objetivo: minimizar la distancia total recorrida por los pacientes.
  \item Establecieron restricciones como:
  \begin{itemize}
    \item Cada paciente debe ir a un solo hospital.
    \item Cada hospital tiene capacidad limitada.
    \item Solo se usan hospitales disponibles.
  \end{itemize}
  \item El modelo fue resuelto con un programa computacional para obtener la mejor asignación.
\end{itemize}
\end{frame}

\begin{frame}{4. ¿Qué resultados obtuvieron?}
\begin{itemize}
  \item Si reorganizaban a los pacientes, se podía ahorrar más de \textbf{293,000 km} de viajes al mes.
  \item Cada paciente viajaría en promedio \textbf{13 km menos}.
  \item Esto significaba más comodidad y ahorro de tiempo y dinero.
\end{itemize}
\end{frame}

\begin{frame}{5. ¿Por qué es importante?}
\begin{itemize}
  \item Menos viajes implican mejor calidad de vida para los pacientes.
  \item La programación lineal ayudó a tomar decisiones justas y eficientes.
  \item Mostró cómo las matemáticas pueden mejorar los sistemas de salud.
\end{itemize}
\end{frame}

\begin{frame}{6. Conclusión}
\begin{itemize}
  \item El estudio demostró que con modelos matemáticos se pueden resolver problemas reales.
  \item Es posible mejorar servicios públicos con lógica y organización.
  \item La programación lineal fue clave para mejorar el acceso al tratamiento croncológico en Brasil.
\end{itemize}
\end{frame}

\end{document}
