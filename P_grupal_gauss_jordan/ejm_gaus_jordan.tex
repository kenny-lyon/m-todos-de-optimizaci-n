\documentclass{beamer}
\usepackage[spanish]{babel}
\usepackage{amsmath}
\usepackage{array}
\usepackage{graphicx}
\usepackage{booktabs}
\usepackage{color}

\usetheme{Berlin}

% ----------- PERSONALIZACIÓN ----------
% Colores personalizados
\definecolor{azulunam}{RGB}{0,53,148}
\setbeamercolor{structure}{fg=azulunam}
\setbeamercolor{block title}{bg=azulunam!20, fg=black}

% Eliminar pie de página (autor, institución, fecha)
\setbeamertemplate{footline}{} 
\setbeamertemplate{navigation symbols}{} % Quita íconos de navegación

% ----------- INFORMACIÓN GENERAL ----------
\title{Método de Gauss-Jordan}
\subtitle{Caso práctico: Producción de cereales}

\begin{document}

% Sección: Planteamiento del problema
\section{Planteamiento del problema}

% Enunciado
\begin{frame}{Planteamiento del problema}
\begin{block}{Enunciado}
\footnotesize
En una empresa de cereales se producen y venden tres tipos de productos: granola ($x$), arroz tostado ($y$) y cereal endulzado ($z$).

Se sabe que:

\begin{itemize}
    \item En el primer trimestre del año se produjeron 16 toneladas de granola, 14 toneladas de arroz tostado y 12 toneladas de cereal endulzado.
    \item En el segundo trimestre, la producción total de los tres cereales fue de 58 toneladas.
    \item En el último trimestre, dos veces la cantidad de granola más tres veces la de arroz tostado fue igual a la cantidad de cereal endulzado fabricado.
\end{itemize}

\textbf{Instrucciones:}
\begin{enumerate}[a.]
    \item Plantee el sistema de ecuaciones.
    \item Resuélvalo utilizando el método de Gauss-Jordan.
\end{enumerate}
\end{block}
\end{frame}

% Tabla de datos organizados
\begin{frame}{Datos organizados}
\begin{block}{Resumen de los datos}
\begin{table}[h]
\centering
\renewcommand{\arraystretch}{1.4}
\begin{tabular}{lcc}
\toprule
\textbf{Trimestre} & \textbf{Producción} & \textbf{Ecuación} \\
\midrule
1er Trimestre & $x=16$, $y=14$, $z=12$ & -- \\
2do Trimestre & $x+y+z=58$ & $x + y + z = 58$ \\
3er Trimestre & $2x + 3y = z$ & $2x + 3y - z = 0$ \\
\bottomrule
\end{tabular}
\end{table}
\end{block}
\end{frame}

% Sistema de ecuaciones
\begin{frame}{Parte a: Sistema de ecuaciones}
\begin{block}{Planteamiento}
Diferencia entre el primer y segundo trimestre:
\begin{align*}
(x_2 - x_1) + (y_2 - y_1) + (z_2 - z_1) &= 58 - (16+14+12) \\
-x + y + z &= 10 \quad \text{(Ecuación 1)}
\end{align*}

Sistema completo:
\begin{align*}
-x + y + z &= 10 \\
x + y + z &= 58 \\
2x + 3y - z &= 0
\end{align*}
\end{block}

\begin{alertblock}{Matriz aumentada}
\centering
$\left[
\begin{array}{ccc|c}
-1 & 1 & 1 & 10 \\
1 & 1 & 1 & 58 \\
2 & 3 & -1 & 0 \\
\end{array}
\right]$
\end{alertblock}
\end{frame}

% Gauss-Jordan: Paso 1
\section{Solución por Gauss-Jordan}
\begin{frame}{Paso 1: Eliminación de $x$}
\begin{block}{Operaciones}
\begin{align*}
R_2 &\leftarrow R_2 + R_1 \\
R_3 &\leftarrow R_3 + 2R_1
\end{align*}
\end{block}

\begin{exampleblock}{Resultado}
\centering
$\left[
\begin{array}{ccc|c}
-1 & 1 & 1 & 10 \\
0 & 2 & 2 & 68 \\
0 & 5 & 1 & 20 \\
\end{array}
\right]$
\end{exampleblock}
\end{frame}

% Paso 2
\begin{frame}{Paso 2: Eliminación de $y$ en $R_3$}
\begin{block}{Operaciones}
\begin{align*}
R_2 &\leftarrow \tfrac{1}{2} R_2 \\
R_3 &\leftarrow R_3 - 5R_2
\end{align*}
\end{block}

\begin{exampleblock}{Resultado}
\centering
$\left[
\begin{array}{ccc|c}
-1 & 1 & 1 & 10 \\
0 & 1 & 1 & 34 \\
0 & 0 & -4 & -150 \\
\end{array}
\right]$
\end{exampleblock}
\end{frame}

% Paso 3
\begin{frame}{Paso 3: Eliminación de $z$}
\begin{block}{Operaciones}
\begin{align*}
R_3 &\leftarrow -\tfrac{1}{4}R_3 \\
R_1 &\leftarrow R_1 - R_3 \\
R_2 &\leftarrow R_2 - R_3
\end{align*}
\end{block}

\begin{exampleblock}{Resultado}
\centering
$\left[
\begin{array}{ccc|c}
-1 & 1 & 0 & -27.5 \\
0 & 1 & 0 & -3.5 \\
0 & 0 & 1 & 37.5 \\
\end{array}
\right]$
\end{exampleblock}
\end{frame}

% Paso 4 y solución
\begin{frame}{Paso 4: Solución final}
\begin{block}{Último paso}
\[ R_1 \leftarrow R_1 - R_2 \]
\end{block}

\begin{alertblock}{Matriz final}
\centering
$\left[
\begin{array}{ccc|c}
-1 & 0 & 0 & -24 \\
0 & 1 & 0 & -3.5 \\
0 & 0 & 1 & 37.5 \\
\end{array}
\right]$
\end{alertblock}

\begin{block}{Solución}
\begin{align*}
x &= 24 \text{ t} \\
y &= -3.5 \text{ t} \\
z &= 37.5 \text{ t}
\end{align*}
\end{block}
\end{frame}

% Verificación
\begin{frame}{Verificación del sistema}
\begin{block}{Sustitución en las ecuaciones}
\begin{itemize}
\item $-x + y + z = -24 + (-3.5) + 37.5 = 10 \quad \checkmark$
\item $x + y + z = 24 + (-3.5) + 37.5 = 58 \quad \checkmark$
\item $2x + 3y - z = 48 + (-10.5) - 37.5 = 0 \quad \checkmark$
\end{itemize}
\end{block}

\begin{alertblock}{Nota importante}
El valor negativo de $y$ indica una posible inconsistencia en los datos reportados. Se recomienda verificar el planteamiento original.
\end{alertblock}
\end{frame}

% Interpretación
\section{Interpretación}
\begin{frame}{Análisis de resultados}
\begin{columns}[T]
\column{0.5\textwidth}
\begin{block}{Valores finales}
\begin{itemize}
\item \textcolor{green}{Granola}: 24 t
\item \textcolor{red}{Arroz tostado}: -3.5 t
\item \textcolor{blue}{Endulzado}: 37.5 t
\end{itemize}
\end{block}

\begin{block}{Causas posibles}
\begin{itemize}
\item Error en los datos
\item Déficit real de producción
\item Mala planificación
\end{itemize}
\end{block}

\column{0.5\textwidth}
\begin{alertblock}{Conclusión}
\begin{itemize}
\item El método Gauss-Jordan permite detectar inconsistencias.
\item Es fundamental verificar los datos iniciales.
\item Resultados negativos deben analizarse cuidadosamente.
\end{itemize}
\end{alertblock}

% Comentado el gráfico
% \includegraphics[width=0.8\linewidth]{grafico_produccion}
\end{columns}
\end{frame}

\end{document}
