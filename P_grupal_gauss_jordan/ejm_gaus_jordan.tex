\documentclass{beamer}
\usepackage[spanish]{babel}
\usepackage[utf8]{inputenc}
\usepackage{amsmath}
\usepackage{array}
\usepackage{graphicx}
\usepackage{booktabs}
\usepackage{color}
\usepackage{ragged2e}
\usetheme{Madrid}

\definecolor{azulunam}{RGB}{0,53,148}
\setbeamercolor{structure}{fg=azulunam}
\setbeamercolor{block title}{bg=azulunam!20, fg=black}

\setbeamertemplate{footline}{} 
\setbeamertemplate{navigation symbols}{} 

\title{Metodo de Gauss-Jordan}
\subtitle{Caso practico: Produccion de cereales}

\begin{document}

\section{Planteamiento del problema}

\begin{frame}{Planteamiento del problema}
\begin{block}{Enunciado}
\footnotesize
En una empresa de cereales se producen y venden tres tipos de productos: granola ($x$), arroz tostado ($y$) y cereal endulzado ($z$).

Se sabe que:
\begin{itemize}
    \item En el primer trimestre del año se produjeron 16 toneladas de granola, 14 toneladas de arroz tostado y 12 toneladas de cereal endulzado.
    \item En el segundo trimestre, la produccion total de los tres cereales fue de 58 toneladas.
    \item En el ultimo trimestre, dos veces la cantidad de granola mas tres veces la de arroz tostado fue igual a la cantidad de cereal endulzado fabricado.
\end{itemize}

\textbf{Instrucciones:}
\begin{enumerate}[a.]
    \item Plantee el sistema de ecuaciones.
    \item Resuelvalo utilizando el metodo de Gauss-Jordan.
\end{enumerate}
\end{block}
\end{frame}

% Aqui va la taabla
\begin{frame}{Datos organizados}
\begin{block}{Resumen de los datos}
\begin{table}[h]
\centering
\renewcommand{\arraystretch}{1.4}
\begin{tabular}{lcc}
\toprule
\textbf{Trimestre} & \textbf{Produccion} & \textbf{Ecuacion} \\
\midrule
1er Trimestre & $x=16$, $y=14$, $z=12$ & -- \\
2do Trimestre & $x+y+z=58$ & $x + y + z = 58$ \\
3er Trimestre & $2x + 3y = z$ & $2x + 3y - z = 0$ \\
\bottomrule
\end{tabular}
\end{table}
\end{block}
\end{frame}

\begin{frame}{Parte a: Sistema de ecuaciones}
\begin{block}{Planteamiento}
\footnotesize
Diferencia entre el primer y segundo trimestre:
\begin{align*}
(x_2 - x_1) + (y_2 - y_1) + (z_2 - z_1) &= 58 - (16+14+12) \\
-x + y + z &= 10 \quad \text{(Ecuación 1)}
\end{align*}

Sistema completo:
\begin{align*}
-x + y + z &= 10 \\
x + y + z &= 58 \\
2x + 3y - z &= 0
\end{align*}
\end{block}

\begin{alertblock}{Matriz aumentada}
\scriptsize
\[
\left[
\begin{array}{rrr|r}
-1 & 1 & 1 & 10 \\
1 & 1 & 1 & 58 \\
2 & 3 & -1 & 0 \\
\end{array}
\right]
\]
\end{alertblock}
\end{frame}

\section{Solucion por Gauss-Jordan}
\begin{frame}{Paso 1: Eliminacion de $x$}
\begin{block}{Operaciones}
\begin{align*}
R_2 &\leftarrow R_2 + R_1 \\
R_3 &\leftarrow R_3 + 2R_1
\end{align*}
\end{block}

\begin{exampleblock}{Resultado}
\scriptsize
\[
\left[
\begin{array}{rrr|r}
-1 & 1 & 1 & 10 \\
0 & 2 & 2 & 68 \\
0 & 5 & 1 & 20 \\
\end{array}
\right]
\]
\end{exampleblock}
\end{frame}

\begin{frame}{Paso 2: Eliminacion de $y$ en $R_3$}
\begin{block}{Operaciones}
\begin{align*}
R_2 &\leftarrow \tfrac{1}{2} R_2 \\
R_3 &\leftarrow R_3 - 5R_2
\end{align*}
\end{block}

\begin{exampleblock}{Resultado}
\scriptsize
\[
\left[
\begin{array}{rrr|r}
-1 & 1 & 1 & 10 \\
0 & 1 & 1 & 34 \\
0 & 0 & -4 & -150 \\
\end{array}
\right]
\]
\end{exampleblock}
\end{frame}

\begin{frame}{Paso 3: Eliminacion de $z$}
\begin{block}{Operaciones}
\begin{align*}
R_3 &\leftarrow -\tfrac{1}{4}R_3 \\
R_1 &\leftarrow R_1 - R_3 \\
R_2 &\leftarrow R_2 - R_3
\end{align*}
\end{block}

\begin{exampleblock}{Resultado}
\scriptsize
\[
\left[
\begin{array}{rrr|r}
-1 & 1 & 0 & -27.5 \\
0 & 1 & 0 & -3.5 \\
0 & 0 & 1 & 37.5 \\
\end{array}
\right]
\]
\end{exampleblock}
\end{frame}

\begin{frame}{Paso 4: Solucion final}
\begin{block}{Último paso}
\[ R_1 \leftarrow R_1 - R_2 \]
\end{block}

\begin{alertblock}{Matriz final}
\scriptsize
\[
\left[
\begin{array}{rrr|r}
-1 & 0 & 0 & -24 \\
0 & 1 & 0 & -3.5 \\
0 & 0 & 1 & 37.5 \\
\end{array}
\right]
\]
\end{alertblock}

\begin{block}{Solucion}
\begin{align*}
x &= 24 \text{ t} \\
y &= -3.5 \text{ t} \\
z &= 37.5 \text{ t}
\end{align*}
\end{block}
\end{frame}

\begin{frame}{Verificacion del sistema}
\begin{block}{Sustitucion en las ecuaciones}
\scriptsize
\begin{itemize}
\item $-x + y + z = -24 + (-3.5) + 37.5 = 10 \quad \checkmark$
\item $x + y + z = 24 + (-3.5) + 37.5 = 58 \quad \checkmark$
\item $2x + 3y - z = 48 + (-10.5) - 37.5 = 0 \quad \checkmark$
\end{itemize}
\end{block}

\begin{alertblock}{Nota importante}
El valor negativo de $y$ indica una posible inconsistencia en los datos reportados. Se recomienda verificar el planteamiento original.
\end{alertblock}
\end{frame}

\section{Interpretacion}
\begin{frame}{Analisis de resultados}
\begin{columns}[T]
\column{0.5\textwidth}
\begin{block}{Valores finales}
\begin{itemize}
\item \textcolor{green}{Granola}: 24 t
\item \textcolor{red}{Arroz tostado}: -3.5 t
\item \textcolor{blue}{Endulzado}: 37.5 t
\end{itemize}
\end{block}

\begin{block}{Causas posibles}
\begin{itemize}
\item Error en los datos
\item Deficit real de produccion
\item Mala planificacion
\end{itemize}
\end{block}

\column{0.5\textwidth}
\begin{alertblock}{Conclusion}
\begin{itemize}
\item El metodo Gauss-Jordan permite detectar inconsistencias.
\item Es fundamental verificar los datos iniciales.
\item Resultados negativos deben analizarse cuidadosamente.
\end{itemize}
\end{alertblock}
\end{columns}
\end{frame}

\end{document}
